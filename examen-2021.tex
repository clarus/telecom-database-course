\documentclass[a4paper,11pt]{article}
\usepackage[utf8]{inputenc}
\usepackage[T1]{fontenc}
\usepackage[english,francais]{babel}
\usepackage{hyperref}
\usepackage{graphicx}
\usepackage{pgf}
\usepackage{tikz}
\usetikzlibrary{arrows,automata}
\usetikzlibrary{er,positioning}

\title{Examen SD-202 bases de données}
\date{Mardi 22 Juin 2021, 8h30-11h30}

\begin{document}
\maketitle
{\large Documents et Internet autorisés.}

Site du cours: \url{https://clarus.github.io/telecom-database-course/}

\section{Questions générales}{5 points}
\begin{enumerate}
  \item Comment est généralement choisi le nombre maximal de fils dans les nœuds d'un arbre B~? Pourquoi choisir une telle valeur~?
  \item Donner deux intérêts à utiliser un système de gestion de bases de données, plutôt que d'utiliser directement des fichiers pour enregistrer ses données.
  \item Quel mot clé utiliser pour filtrer les résultats d'une requête agrégée~?
  \item Pourquoi n'est-il généralement pas possible de modifier les données dans une vue~? Donner un exemple illustrant la difficulté.
  \item Donner un exemple de différence conceptuelle entre l'algèbre relationnelle ensembliste et le langage SQL.
\end{enumerate}

\section{Algèbre relationnelle}{1 point}
\begin{enumerate}
  \item Exprimer la requête suivante~:\\
    \begin{tabular}{l}
      SELECT DISTINCT NumAccident\\
      FROM Accident, Véhicule\\
      WHERE\\
      \quad Accident.NumVéhicule = Véhicule.NumVéhicule AND\\
      \quad Véhicule.Cylindrée > 6\\
    \end{tabular}\\
    en utilisant les opérateur de l'algèbre relationnel suivants~:
    \begin{itemize}
      \item sélection~$\sigma$
      \item projection~$\pi$
      \item jointure sur une colonne~$\bowtie_c$
    \end{itemize}
\end{enumerate}

\section{Relations fonctionnelles}{7 points}
\begin{enumerate}
  \item Démontrer la règle de transitivité, en revenant à la définition d'une dépendance fonctionnelle~:\\
  \emph{Si $A \rightarrow B$ et $B \rightarrow C$ alors $A \rightarrow C$.}
  \item Comment représenter une liste (non-ordonnée) en première forme normale~? Illustrer en donnant un exemple de clients pouvant avoir plusieurs numéros de téléphone.
  \item La forme BCNF implique-t-elle les formes 2NF et 3NF~?
  \item
    Soit une relation sur les attributs~:
    \[
      \begin{array}{l}
        \mathrm{Adresse}, \mathrm{Cat}, \mathrm{Compte}, \mathrm{Datecom}, \mathrm{Libelle}, \mathrm{Localite}, \mathrm{NCli},\\
        \mathrm{NCom}, \mathrm{Nom}, \mathrm{NPro}, \mathrm{Prix}, \mathrm{QCom}, \mathrm{QStock}
      \end{array}
    \]
    munie des dépendances fonctionnelles suivantes~:
    \begin{itemize}
      \item $\mathrm{NCli} \rightarrow \mathrm{Nom}, \mathrm{Adresse}, \mathrm{Localite}, \mathrm{Cat}, \mathrm{Compte}$
      \item $\mathrm{NPro} \rightarrow \mathrm{Libelle}, \mathrm{Prix}, \mathrm{QStock}$
      \item $\mathrm{NCom} \rightarrow \mathrm{NCli}, \mathrm{Datecom}$
      \item $\mathrm{NCom}, \mathrm{NPro} \rightarrow \mathrm{QCom}$
    \end{itemize}
    Décomposer cette relation en forme BCNF.
  \item Sur les tables issues de la décomposition en forme BCNF de la question précédente, écrire les requêtes SQL permettant d'obtenir~:
    \begin{itemize}
      \item la liste des localités pour lesquelles il existe au moins un client,
      \item les localités des clients qui commandent le produit de numéro "FOOBAR",
      \item la somme des prix de tous les articles en stock,
      \item les localités des clients qui ont passé au moins une commande,
      \item les localités dont aucun client n'a passé de commande.
    \end{itemize}
  \item Proposer une couverture minimale~\emph{(minimal cover)} pour l'ensemble de dépendances fonctionnelles suivant~:
    \[
      \{D \rightarrow B, BE \rightarrow C, DA \rightarrow D, C \rightarrow F, DE \rightarrow F, FGH \rightarrow C, A \rightarrow B, AD \rightarrow G\}
    \]
    On suppose que les attributs sont les lettres $A$, $B$, \dots, $H$.
\end{enumerate}

\section{Modèle entité-association}{7 points}
\begin{enumerate}
  \item
    Étant donné le schéma relationnel suivant~:
    \begin{itemize}
      \item Dossier(\underline{NumDossier}, Titre, DateEnreg, \#NomDirection, \#NomDepart, \#NomService)
      \item Service(\underline{NomService}, Responsable, \#NomDpart)
      \item Employe(\underline{NumEmp}, NomEmp, Adresse, \#NomService)
      \item Departement(\underline{NomDepart}, Localisation, \#NomDirection)
      \item Direction(\underline{NomDirection}, President, Adresse)
    \end{itemize}
    donner un diagramme entité-association correspondant. On suppose que les attributs préfixés par~\# correspondent à des clés étrangères.
  \item Donner une extension du diagramme telle que la date d'arrivée d'un employé dans un service soit présente.
  \item Donner un schéma relationnel correspondant au modèle entité-association suivant~:\\\\
    \begin{tikzpicture}[auto,node distance=1.5cm]
      \node[entity] (keyword) {
        \begin{tabular}{l}
          \textbf{Mot-clé}\\
          \hline
          \underline{Valeur}
        \end{tabular}
      };
      \node[relationship] (describe) [right = of keyword] {Décrit};
      \node[entity] (ouvrage) [right = of describe] {
        \begin{tabular}{l}
          \textbf{Ouvrage}\\
          \hline
          \underline{NumOuvr}\\
          Titre\\
          Auteurs\\
          Éditeur
        \end{tabular}
      };
      \node[relationship] (de) [below = of ouvrage] {De};
      \node[entity] (exemplaire) [below = of de] {
        \begin{tabular}{l}
          \textbf{Exemplaire}\\
          \hline
          \underline{NumEx}\\
          Position\\
          DateAchat
        \end{tabular}
      };
      \node[relationship] (emprunte) [right = of exemplaire] {Emprunte};
      \node[entity] (emprunteur) [above = of emprunte] {
        \begin{tabular}{l}
          \textbf{Emprunteur}\\
          \hline
          \underline{NumEmpr}\\
          NomEmpr\\
          Adresse
        \end{tabular}
      };
      \path (describe)
        edge node {$0$..$\star$} (keyword)
        edge node {$0$..$\star$} (ouvrage);
      \path (de)
        edge node {$0$..$\star$} (ouvrage)
        edge node {$1$..$1$} (exemplaire);
      \path (emprunte)
        edge node {$0$..$1$} (exemplaire)
        edge node {$1$..$\star$} (emprunteur);
    \end{tikzpicture}
  \item
    Écrire les requêtes SQL permettant d'obtenir~:
    \begin{itemize}
      \item le nombre d'occurences de chaque mot clé,
      \item les exemplaires correspondant à un mot clé donné,
      \item les personnes ayant emprunté un ouvrage de chaque éditeur,
      \item la personne ayant le plus d'exemplaires différents d'un même ouvrage.
    \end{itemize}
\end{enumerate}

\end{document}
